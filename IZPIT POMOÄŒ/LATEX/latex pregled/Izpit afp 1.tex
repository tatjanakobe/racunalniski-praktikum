\documentclass[a4paper,12pt]{article}
\usepackage[slovene]{babel}
\usepackage{amsfonts,amssymb,amsmath,mathrsfs,amsthm}
\usepackage[utf8]{inputenc}
\usepackage[T1]{fontenc}
\usepackage{url}
\usepackage{hyperref}

{\theoremstyle{definition}
\newtheorem{zgled}{Zgled}}

\newcommand{\B}{\mathscr{B}}

%%%%%%%%%%%%%%%%%%%%%%%%%%%%%%%%%%%%%%%%%%%%%%%%%%%%%%%%%%%%%%%%%%%%%%%%%%%%%

\def\R{\mathbb{R}}  % mnozica realnih stevil

%%%%%%%%%%%%%%%%%%%%%%%%%%%%%%%%%%%%%%%%%%%%%%%%%%%%%%%%%%%%%%%%%%%%%%%%%%%%%

% Thomaeova funkcija
% Beno Učakar
\title{Thomaeova funkcija}
\author{Beno Učakar}
\date{}



\begin{document}

\maketitle

Oglejmo si Thomaeovo funkcijo, ki je primer funkcije prvega Bairovega razreda.
Take funkcije lahko definiramo na naslednji način.
Naj bo $D \subseteq \mathbb{R}$. Funkcija $f : D \rightarrow \mathbb{R}$ je \emph{funkcija prvega Bairovega razreda}, 
če obstaja funkcijsko zaporedje $\{f_n\}$ zveznih funkcij na $D$, ki po točkah konvergira k $f$. 
Ta razred označimo z $\B(D)$ oziroma, če ne bo nevarnosti zmede, kar z $\B$.

% začetek zgleda
   
   \begin{zgled}
     Funkcijsko zaporedje $f_n : (0,1)\rightarrow\mathbb{R}$ definiramo na sledeč način.
     Za vsak $p,q \in \mathbb{N}_0$, $1 \le q < n$ in $0 \le p \le q$ definiramo
     \begin{itemize}
         \item \(f_n(x) = \max\left\{\frac{1}{n}, \frac{1}{q} + 2n^2\left(x - \frac{p}{q}\right)\right\}\) na intervalu \(\left(\frac{p}{q} - \frac{1}{2n^2}, \frac{p}{q}\right)\) in
         \item \(f_n(x) = \max\left\{\frac{1}{n}, \frac{1}{q} - 2n^2\left(x - \frac{p}{q}\right)\right\}\) na intervalu \(\left(\frac{p}{q}, \frac{p}{q} + \frac{1}{2n^2}\right)\).
     \end{itemize}
     V vseh ostalih točkah naj bo $f_n(x) = \frac{1}{n}$.
     Preverimo lahko, da so intervali $\left(\frac{p}{q} - \frac{1}{2n^2}, \frac{p}{q} + \frac{1}{2n^2}\right)$ paroma disjunktni in zgornja definicija je dobra.
     Opazimo, da je $f_n(x)$ odsekoma linearna zvezna. Če vzamemo limito po točkah, dobimo
 
     \begin{align*}
        f(x,y) = \begin{cases}
            \frac{1}{q}; & x = \frac{p}{q}   \text{ je pokrajšan ulomek za }   p,q \in \mathbb{N} \\
            0; & x   \text{ je racionalen }
        \end{cases}
     \end{align*}
 
     Pokazali smo, da \emph{Thomaeova funkcija} pripada $\B$. 
   \end{zgled}
% konec zgleda

%%%%%%%%%%%%%%%%%%%%%%%%%%%%%%%%%%%%%%%%%%%%%%%%%%%%%%%%%%%%%%%%%%%%%
\nocite{*}
\bibliographystyle{siam}
\bibliography{viri}

\end{document}